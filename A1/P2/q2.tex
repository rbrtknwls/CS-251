% <- percent signs are used to comment
\documentclass[12pt]{article}

%%%%%% PACKAGES - this part loads additional material for LaTeX %%%%%%%%%
% Nearly anything you want can be done in LaTeX if you load the right package 
% (search ctan.org or google it if you are looking for something).  We will load
% here a few that we need for this document or that we expect you to need later.

% The next 3 lines are needed to fix shortcomings of TeX that only make sense given its 40-year history ...
% Simple keep and ignore.
\usepackage[utf8]{inputenc}
\usepackage[T1]{fontenc}
\usepackage{lmodern}
\usepackage{amsmath}
\usepackage{changepage}
\usepackage{lipsum}
\usepackage{bm}
\usepackage{ulem}


% Custom margins (and paper sizes etc.) because LaTeX else wastes much space
\usepackage[margin=1in]{geometry}

% The following packages are created by the American Mathematical Society (AMS)
% and provide lots of tools for special fonts, symbols, theorems, and proof
\usepackage{amsmath,amsfonts,amssymb,amsthm}
% mathtools contains many detail improvements over ams and core tex
\usepackage{mathtools}

% graphicx is required for images
\usepackage{graphicx}

% enumitem used for customizing enumerations
\usepackage[shortlabels]{enumitem}

% tikz is the package used for drawing, in particular for drawing trees. You may also find simplified packages like tikz-qtree and forest useful
\usepackage{tikz}

% hyperref allows links, urls, and many other PDF tricks.  We load it here
%          in such a way that the PDF file has info about it
\usepackage[%
	pdftitle={CS251 Assignment 0},%
	hidelinks,%
]{hyperref}


%%%%%% COMMANDS - here you can define your own LaTeX-commands %%%%%%%%%

%%%%%% End of Preamble %%%%%%%%%%%%%

\begin{document}

\begin{center}
{\Large\textbf{CS251, Spring 2022}}\\
\vspace{2mm}
{\Large\textbf{Assignment 1: Question 2}}\\
\vspace{3mm}
\end{center}
\begin{adjustwidth}{0em}{0pt}
Consider the following ARM program (where PC starts at 0):
\begin{align*}
    \begin{aligned}
       000& \ \text{ADD X0, X31, X31  \ \ ; Initialize X0 as 0} \\
       004& \ \text{ADDI X1, X31, \#64 \ ; Initialize X1 as 64} \\
       008& \ \text{LDUR X2, [X1, \#0] \ ; Load into X2 the value at X1} \\
       012& \ \text{ADDI X2, X2, \#1  \ \ \ \ ; Add 1 (-1 becomes 0)} \\
       016& \ \text{CBZ X2, \#7 \ \ \ \ \ \ \ \ \ \ \ \ ; Exit loop if value is 0} \\
       020& \ \text{ADDI X1, X1, \#8 \ \ \ \ \ ; Move to next value (8 spots in memory)} \\
       024& \ \text{LDUR X2, [X1, \#0] \ \ \ ; Load into X2 the value at X1} \\
       028& \ \text{CBNZ X2, \#2 \ \ \ \ \ \ \ \ \ \ ; Skip Increment if value is non zero} \\
       032& \ \text{ADDI X0, X0, \#1 \ \ \ \ \ ; Increment counter} \\
       036& \ \text{ADDI X1, X1, \#8 \ \ \ \ \ ; Move to next value (8 spots in memory)} \\
       040& \ \text{B \#-8 \ \ \ \ \ \ \ \ \ \ \ \ \ \ \ \ \ \ \ \ \  ; Loop to start} \\
       044& \ \text{STUR X0, [X31, \#56] \ ; Store counter} \\
       048& \ \\
       052& \  \\
    \end{aligned}
\end{align*}
\end{adjustwidth}



\end{document}