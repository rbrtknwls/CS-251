% <- percent signs are used to comment
\documentclass[12pt]{article}

%%%%%% PACKAGES - this part loads additional material for LaTeX %%%%%%%%%
% Nearly anything you want can be done in LaTeX if you load the right package 
% (search ctan.org or google it if you are looking for something).  We will load
% here a few that we need for this document or that we expect you to need later.

% The next 3 lines are needed to fix shortcomings of TeX that only make sense given its 40-year history ...
% Simple keep and ignore.
\usepackage[utf8]{inputenc}
\usepackage[T1]{fontenc}
\usepackage{lmodern}
\usepackage{amsmath}
\usepackage{changepage}
\usepackage{lipsum}
\usepackage{bm}
\usepackage{ulem}


% Custom margins (and paper sizes etc.) because LaTeX else wastes much space
\usepackage[margin=1in]{geometry}

% The following packages are created by the American Mathematical Society (AMS)
% and provide lots of tools for special fonts, symbols, theorems, and proof
\usepackage{amsmath,amsfonts,amssymb,amsthm}
% mathtools contains many detail improvements over ams and core tex
\usepackage{mathtools}

% graphicx is required for images
\usepackage{graphicx}

% enumitem used for customizing enumerations
\usepackage[shortlabels]{enumitem}

% tikz is the package used for drawing, in particular for drawing trees. You may also find simplified packages like tikz-qtree and forest useful
\usepackage{tikz}

% hyperref allows links, urls, and many other PDF tricks.  We load it here
%          in such a way that the PDF file has info about it
\usepackage[%
	pdftitle={CS251 Assignment 0},%
	hidelinks,%
]{hyperref}


%%%%%% COMMANDS - here you can define your own LaTeX-commands %%%%%%%%%

%%%%%% End of Preamble %%%%%%%%%%%%%

\begin{document}

\begin{center}
{\Large\textbf{CS251, Spring 2022}}\\
\vspace{2mm}
{\Large\textbf{Assignment 1: Question 4}}\\
\vspace{3mm}
\end{center}
\textbf{Q4a)} Give a truth table for the circuit: 

\begin{center}
\begin{tabular}{||c c | c | c | c | c | c |||} 
 \hline
 X & Y & A & $\neg$ A & B & C & D \\ [0.5ex] 
 \hline\hline
 0 & 0 & 0 & 1 & 0 & 0 & 0 \\ 
 \hline
 0 & 1 & 1 & 0 & 0 & 1 & 1 \\
 \hline
 1 & 0 & 1 & 0 & 0 & 1 & 1 \\
 \hline
 1 & 1 & 1 & 0 & 0 & 1 & 1 \\ [1ex] 
 \hline
\end{tabular}
\end{center}

\begin{adjustwidth}{0em}{0pt}
\textbf{Q4b)} Give the boolean formula for D:
\[ D = \overline{X}Y + X\overline{Y} + XY \]
\end{adjustwidth}
\begin{adjustwidth}{0em}{0pt}
\textbf{Q4c)} Simplify the following Boolean formula:
\begin{align*}
    \begin{aligned}
       F &= XYZ + \overline{X}Z + XZ + YZ \\
         &= Z( YZ + \overline{X} + X + Y) \text{ (Distributive Law) } \\
         &= Z( YZ + 1 + Y)                \text{ (Inverse Law) }\\
         &= Z(1)                \text{ (One Law) } \\
         &= Z
    \end{aligned}
\end{align*}
\end{adjustwidth}




\end{document}